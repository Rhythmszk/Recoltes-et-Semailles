\documentclass[oneside,12pt]{book}
\usepackage[utf8]{inputenc}
\usepackage[UTF8]{ctex}
\usepackage{biblatex}
\usepackage{amssymb}
\usepackage{latexsym}
\usepackage{amsmath}
\usepackage{cases}
\usepackage{geometry}
\usepackage{graphicx}
\usepackage{float}
\usepackage{listings}
\usepackage{enumerate}
\usepackage{color}
\newcommand{\memo}[1]{\color{red}#1}

\addbibresource{bib.bib}
% \setlength{\parindent}{0em}
\bibliography{bib}
\geometry{a4paper}
% \geometry{a4paper,scale=0.8}
\linespread{1.5}

\begin{document}

\begin{titlepage}
\begin{center}
\linespread{1.2}~\\[1cm]
\linespread{1}
% \includegraphics[width=10cm]{name.pdf}\\[3cm]
\linespread{1.9}\huge {\bfseries 收获与播种}\\[2.5cm]
\linespread{1.9}\Large {\bfseries ----- 对数学家过往的思考\footnotemark 与见证}\\[2.5cm]
\footnotetext{\memo{La traduction des mots "r\'eflexion" et "t\'emoignage" ?}}

\Large 格罗滕迪克~著\\[0.3cm]
\Large fdzkzhou 试译\\[4cm] 
\large\today
\end{center}

\newpage
\thispagestyle{empty}
~\\[4cm]
\begin{center}
    \large \hspace{3cm} 致父母
\end{center}
\end{titlepage}

\frontmatter

\tableofcontents


\mainmatter

\section{作为序言……}

\section{漫游穿过一件作品 ----- 或者说孩子与母亲}

\subsection{事物的魔力}

\subsection{独处的重要性}

\subsection{内心的冒险 ----- 幻想与见证}

\section{一封信}

\subsection{千页的信}

\hfill 1985年5月

\noindent 1. 我传给你的这些文字,由我的大学细心打印出了有限份样本,它不是单行本,也不是预印本。它的名字,收获与播种,很清楚地说明了这点。我把它寄给你就好像寄一封长长的信 ----- 一封非常私人的信。要是我寄给了你,而非满足于你总有一天会通过书店里卖的书(如果有出版商冒这个险的话)里了解到的(如果你好奇的话),那是因为比起跟别人讲,我更愿意跟你讲。在写这些的时候我不止一次想到你 ----- 必须得说这封信我写了一年多了,并且整个人都投入到其中。这是我赠与你的礼物,而且我写的时候很细心地予你我(每个时刻)曾拥有的能赠与的最好的东西。我不知道这个礼物是否会被接受 ----- 你的答复(与否)会让我明白的……

与此同时,我也将《收获与播种》发给了我数学界所有的同事、朋友们和(前)学生们,我与他们曾在某些时刻有过紧密接触,或者他们曾以这种或那种方式出现在我的思考\footnotemark 当中,或指名道姓地或否。有可能你就出现在我的思考中,而且你要是用心而不是只用眼睛和脑袋读的话,肯定能在我没点出你名字的地方认出你自己来。我还把《收获与播种》寄给了其他一些做或者不做科学研究的朋友。
\footnotetext{译注:这里“思考”(réflexion)对应副标题中的“思考”(réflexion)二字。}

你现在在读的,这封告知你并向你呈上“千页的信”(作为前菜……)的“前言信”,也起到序言的作用。写这几行字的时候,序言还没写。另外,《收获与播种》总共有五章(不包括额外的前言)。我这次寄给你第一部分(自负与更新),第二部分(葬礼\footnotemark(1) ----- 或着说中国皇帝的长袍),和第四部分(葬礼(3) ----- 或者说四运算)\footnotemark。这些我觉得和你关系比较大。第三部分(葬礼(2) ----- 阴阳的秘诀)无疑是我见证中最私人的部分,同时也比其他部分更具有超出写成之时所处场合的“普遍”价值。在第四部分(四运算)里我会时常涉及这部分的内容,但第四部分还是可以单独地看,甚至(很大程度上)可以独立于前三部分阅读\footnotemark。如果读我发的这些能让你答复我(这是我所期望的)并且使你想读剩下的部分,要告诉我。我会很高兴把它们寄给你,只要你的答复让我感受到你是真的感兴趣而非表面上的好奇。
\addtocounter{footnote}{-2}
\footnotetext{\memo{La traduction chinoise du mot Enterrement, laquelle sera-t-elle?}}
\addtocounter{footnote}{1}
\footnotetext{我把那些在我的反思中因为种种原因出现、私下里我却不认识的同事单独拎出来。我只给他们寄“四运算”(这和他们最相关),还有“第零卷”,包括这封信和《收获与播种》的前言(加上整个前四部分的详细目录)。}
\addtocounter{footnote}{1}
\footnotetext{通常,你会注意到(“自负与更新”)里的每一“节”或者接下来三部分的每个“note”都自成一体。他们可以独立于其余部分阅读,就好像我们可以饶有兴致地看一只手、一只脚、一根手指或一根脚趾或者整个身体其他大大小小的部分,却不会忘记这是某个整体的一部分,而正是这个(隐含的)整体赋予了这各部分所有的意义。}


\subsection{《收获与播种》的诞生 ----- 迅速的回顾}

\noindent 2. 在这封信前信中,我现在想在几页纸中(如果可能的话)告诉你《收获与播种》讨论的是什么问题 ----- 尽可能详尽地,不像那唯一的副标题“对数学家过往的思考与见证”(你应该猜到这是指我的过往)。《收获与播种》里有很多东西,不同人将会看到不同的东西:一段探索过往的\emph{旅程};对存在性的\emph{沉思};对一个环境或一个时代\emph{社会风俗的描绘}(或者对时代变迁中潜在的、无法避免的转变的描绘……);一个\emph{调查}(有时像刑侦调查,有时像本关于数学巨大城邦中贫民窟的武侠小说);\emph{泛滥的数学用语}(它将播种\footnotemark 不止一……);一部应用心理分析学的实践性论著(或者也可以说,一本“\emph{心理分析虚构小说}”);对\emph{自我认知}的赞颂\footnotemark;“我的\emph{忏悔}”;个人\emph{日记};\emph{探索和发现}的心理学;一段\emph{控诉}(无情的,仿佛就应该这样……),甚至是“美丽的数学世界里”的(毫不留情的……)\emph{清算}。可以确定的是,写这本书的时候我没有任何一个时刻是觉得无聊的,而我从中看到了所有的色彩。如果你能在重要的事务中抽空读读这个,要是你读着感到无聊我会很惊讶的。当然除非逼着你读,谁知道呢……
\addtocounter{footnote}{-1}
\footnotetext{译注:此处“播种”(semer)与泛滥(divagation也可以指河水泛滥)呼应,同时也呼应书名中的“播种”(semaille)一词。}
\addtocounter{footnote}{1}
\footnotetext{译注:原文为panégytique,此处按panégyrique(赞颂)译?}

显然这不是只给数学家看的。确实有些地方,它更多的是写给数学家看的。在这封“收获与播种之信”的信前信中,我想总结并重点说一下那些,啊对,那些可能与作为数学家的你尤为相关的内容。为此,最自然的方式就是直接告诉你我是如何慢慢地、一点点地写下这前面所说的四或五块文“砖”的。

如你所知,因为我所属机构(IHES\footnotemark)军方资助的那些事,我在1970年离开了数学的“大世界”。
经过了几年的“文化革命”式的反军国主义和环保主义斗争,这你肯定有所耳闻\footnotemark,
然后我就消失了,隐匿于一座上帝才晓得的外省大学里。有传言说我在养羊、钻井。事实上,除了从事许多其他的职业,我会勇敢地,像大家一样,在大学教书(那曾我并不独特的谋生手段,现在仍是)。有时候我会花几天,甚至几周或者几个月,像个醉鬼一样疯狂地重新做些数学 ----- 我有个纸箱装满了我的那些涂鸦,那些可能只有我才能解读。但这些和我一直以来做的事非常不同,至少初看起来是这样的。1955至1970年间,我偏爱的主题是上同调,具体来说,是各种簇(尤其是代数簇)的上同调。我觉得在这个方向已经做了足够多的东西,以至于其他人没有我也可以解决问题,而且如果要做数学的话,也该换方向了……
\addtocounter{footnote}{-1}
\footnotetext{译注:为Institut des Hautes Études Scientifiques(法国高等科学研究院)的首字母缩写。}
\addtocounter{footnote}{1}
\footnotetext{\memo{avoir quelque echo或许应为“有所共鸣”?}}

1970年,我的生活中有了新的热情,像我以前对数学的热情一样强烈,但不管怎样两者还是很相似的。这是对我称之为“沉思”(因为总是需要给事物取个名称)的热情。这个名称,就像这里会出现的其他名称一样,不可避免地会引起不计其数的误解。就像数学里一样,这需要探索发现。我会在《收获与播种》中经常提到它。无落如何,显然,这才是我直到临终前会一直关心的东西。而且实际上我不止一次认为,数学是过去的事了,而从此,我将会忙于更加严肃的事情 ----- 我将“沉思”。

不过我最终还是意识到(四年前的时候)我的数学热情并没有因此熄灭。甚至,虽然我也不知道为什么而且让我感到惊讶的是,我这个(大约十五年)没想再在生命中发表哪怕一行数学的人,我发现自己突然就开始写一本数学著作了,它没有写完,并且会一卷接着一卷的出来;只要我还在写,我会把我曾经想到的关于数学的东西一股脑写出来,写成一系列(无穷本的\footnotemark)书,可能叫做《数学的思考》吧,我们不谈这个了。
\footnotetext{译注:série infinie既作“无穷(本书)的系列”也作数学术语“无穷级数”。}

两年前,1983年春天时,我正忙于写《追寻champs》\footnotemark(的第一卷),它应当属于《(数学的)思考》第一卷,为的是反思一些我所遇到的事。九个月后,如同预期的那样,这第一卷完成了,也就是说只剩下写前言、重读一遍、做些注记 ----- 然后拿去印刷……
\footnotetext{译注:这里champ指的不是物理里的“场”,而是数学里一个概念,英语译作stack。由于暂时找不到合适的中文译名,故保留法语原词。这是格罗滕迪克探讨champ这一概念的一部数学文集。}

这卷到现在还没写完 ----- 这一年半以来它丝毫未动。还没写完的前言部分已经超过一千二百页了(打字机版),完全写完时会有一千四百页吧。你应该已经猜到了,这所谓的“前言”正是《收获与播种》。最新的想法是,它的内容应该构成所预期的著名“系列”的第一和第二卷加上第三卷的一部分。于是这系列就改名为“思考”(就这么简短,不一定是数学上的)。第三卷剩下的部分主要是一些数学的文章,目前对我来说它们和《追寻champs》一样难弄。这后一本书要等到明年了,因为还有注记、索引,当然,还有前言……

第一幕终!

\subsection{老板的去世 ----- 废弃的工地}

\noindent 3. 我感觉是时候作出一些解释了:为什么我突然退出曾自在地度过生命中二十年的圈子;为什么虽然在没有我的十五年里人们过得不错,而我却有奇怪的“回归”的想法(就像个幽灵\footnotemark);还有为什么一部六七百页的数学著作的前言可以写到一千两百(或四百)页。这里我会进入事情的关键,所以肯定会让你感到难过(不好意思!),甚至让你感到不愉快。因为就像我曾经那样,没有人怀疑你热衷于“乐观”地看待你所处的环境,那里你有你的位置、你的姓名所有这些东西。我知道这是……所以讲到这些地方可能会刺痛你……
\footnotetext{译注:此处“幽灵”(revenant)本义为“(没预料到回归却)回归的人”。}

在《收获与播种》里我处处会提到我离开的事情,可能不太停得下来。这“离开”在我数学生涯中就像是一次重要的休学 ----- 我数学生涯中的事件依据这个“时间点”被分为“之前”或“之后”。当时需要一个强大力量的\emph{冲击}才能将我深深根植的环境中剥离开来,并引到一条清晰的“轨道”上去。这力量来自于与我曾深深认同的环境中某种形式的腐败\footnotemark 的正面交锋,在那之前\footnotemark 都睁只眼闭只眼(克制自己不去参加)。随着时间过去,我意识到这事件之上有一种更深层次的力量作用于我。那就是强烈的\emph{自我更新的需求}。这样的更新我们是无法在一所机构温热舒适的科学暖炉中得以完成和追寻到的。我已有二十年密集的数学活动和大量的数学投入 ----- 同时也是二十年的精神上的停滞不前,“与世隔绝”……没有意识到的时候,我感到窒息 ---- 我想要的是海边的空气!我恰好的“离开”标志着我长期的停滞突然结束了,它也是我向内心深处力量之平衡迈出的第一步,而此前我的内心正被折叠、拧束、固定于一种极度的非平衡态……这次离开可以算的上是\emph{重新出发}\footnotemark ----- 一段全新旅程的开始。
\addtocounter{footnote}{-2}
\footnotetext{这里指的是各个国家全体科学家同军事机器毫无保留地合作,意为建立“寡头社会团体”,比如有合适的经济来源、荣耀和权力。这个问题在《收获与播种》里不怎么触及,就一两次,比如在去年4月2日的“Le respect”笔记里(编号179,第1221-1223页)。}
\addtocounter{footnote}{1}
\footnotetext{译注:原文jusque l'à疑应为jusque là。 }
\addtocounter{footnote}{1}
\footnotetext{译注:“离开”(départ)同时具有“出发”的含义,也可以将其整合为“离去”一词。}

就像我说过的,我的数学热情并没有就此熄灭。它表现于随机的思考中、于同我“之前”所经过的道路中截然不同的道路中。至于我留下的“作品”,“以前”的那些,无论是已经白底黑字印出来还是那些,可能这些更为关键,那些还没有记录下来或出版的 ----- 实际上我感觉,它们似乎已经脱离于我了。去年以前,写《收获与播种》的时候,我未曾考虑过把各种时刻散乱的想法“记录”下来,哪怕只是一点点。我明白我所做的数学工作,尤其是1955至1970年做“几何”的那段时间,是\emph{应当}做的 ----- 而且那些我看见或瞥见的东西,是\emph{应当}出现的,是\emph{应该}在某个重要的日子发表的。还有,我所做的还有那些我让别人做的工作,是好的工作,是我全身心投入其中的工作。我往其中投入了我所有的力气与热爱,而且(我觉得)它从此自成一体 ----- 它已经是个充满生机与活力的事物,不再需要我的呵护照料了。从这方面来看,我得以平静地离开。我留下的这些写或没写的东西,我相信已经把它们交给了正确的人,他们会确保它们依据自身的特性如同有生命的物体一样得以发扬、成长、繁盛。

在这十五年密集的数学工作中,一种广阔的\emph{统一化的视角}在我心里开花、结果和成长,它体现于一些简单的中心思想之中。这是一种“算数几何”的视角,它集合了拓扑、几何(代数几何或解析几何)和算术,我在Weil猜想中看到了它的雏形。这正是我那些年的主要灵感来源,借此我得以梳理出新式几何学的主导思想,并构造出其中一些重要工具。这视角和这些中心思想对我而言已然成为第二天性。(而且在与它断绝联系将近十五年之后,我发现自己现在仍拥有这“第二天性”!)它们于我是如此简单,如此显然,以至于自然而然“大家”都逐渐将其吸收,像我一样。仅仅是在最近,在过去的几个月里,我才意识到这些曾一直引导我的视角或是“中心思想”并没有完整地记录在任何已出版作品之中,顶多隐藏于字里行间。特别是,我宣扬的这个视角,以及它所蕴含的中心思想,即便是完全成熟的二十年后,如今仍为大家所忽视。我有幸发现的这些食物,它们仅存在于我,一名工人\footnotemark,
它们的仆人,它们仅存在于我的心中。
\footnotetext{译注:原文ourvier,应为ouvrier(工人)。}


我所创造的或这或那的工具,到处被用来“打碎”某个困难的著名问题,就像我们用蛮力打开保险箱一样。工具显然是坚固的。然而,我知道它有区别于撬棒的另一种“力量”。它是某个整体的一部分,就像一条肢臂是人体的一部分 ----- 它所来源于的这个整体赋予它意义,它从整体获得生命力。你可以用一根骨头(如果它够大)来打碎颅骨,这是延伸出来的功能。
但这不是它真正的功能、它存在的意义。这些大家相互争夺的工具,我将它们略微视作骨头,它们被细心地切成块、清理,它们可能是人们从一具身体上撕扯下来的 ----- 一具活生生的身体,而他们却假装忽视这点……

我深思熟虑、反复斟酌后所说的这些,是通过各种方式体会到的,比如随着时间的推移、借由那些尚未于思想中成型或尚未以画面的形式被感知的想法,还有通过良好组织过的语言。我已经决定不与这些旧事扯上关系了。那些从越来越遥远的地方传来我这儿的声音,即使已被过滤,却仍具有说服力,只要我稍微停下来一点听它们。我自认为是一个工人,全身心忙活于五六个“工地”\footnotemark ----- 可能算得上最有经验的工人,也可以说是一个在合适的接班人到来前曾长时间孤身劳作于这些地方的长者;一个长者,好吧,但实则与他人并无差别。然后一旦这个工人走了,就会像一家砖砌公司在老板突然去世后宣布倒闭一样:可以说工地次日就被废弃了。“工人们”走了,每人都揣着一些将来可能用得上的东西。钱也没了,于是再也没理由费力干活了……
\footnotetext{在此我通过废弃的“工地”进行阐述,在大概三个月前写的“荒芜的工地”一节(编号176’ 至 178)里我还会重提这一点。一年前,在写葬礼那部分之前,这已经在“我的孤儿们”所含的笔记(编号46)中被提到了,那是我重拾这本作品时写的第一篇,\memo{et sur le sort qui a été le sien}。}

这就是我经过一年多的思考和研究得以理清的一条阐述。但肯定的是,这东西在“某个时刻”已经可以观察到了,从我离开后的前几年开始。抛开Deligne关于Frobenius特征值的绝对值的工作(就像我最终所理解的那样,这是“声望的问题”……)不说 ——- 当我好久才见一次我昔日的好友时,我曾和他们在共事于同个工地,当我问他们“然后呢……?”的时候,他们呈现给我的总是同样的雄辩的姿态,手臂在空中挥舞,像是在请求恩惠……显然,大家都忙于那些比我所感兴趣的东西更加重要的事 ——- 显然的还有,大家都看似忙于重要的事务,但没有什么大成果。实质已经消失了 ——- 是\emph{整体}赋予他们各司其职的意义,我觉得还有\emph{热情}。剩下的他们做的只是把时间浪费在脱离整体的那些琐事之上,各占一角搞自己的玩意儿,或是勉强弄出些成果。

发现这一切都干净地结束之时我确实感到痛苦,虽然我也很想保护自己免于这些痛苦;但毕竟不再能听到别人提及motifs、topos、六运算、de Rham系数、Hodge系数、能把它们联系在一起的“神秘函子”,类似的还有和de Rham系数相关的$\ell$-adic系数、cristaux(如果不是为了了解它们始终是同个点)、“标准猜想”还有其他一些我整理出来的、有证据表明其重要性的问题。同样还有以(在Dieudonné不知疲倦的协助下写的) Éléments de Géométrie Algébrique为首的篇幅巨大的基础工作,它应当维持现有的冲劲继续向前,但却被丢弃在了一旁:所有人都满足于置身于别人耐心组装、搭建、建造的墙和家具之中。工人走了,别人希望他再次撸起袖子拿起灰匙把数不清的没建完的建筑或是\emph{房子}建完,他却不会因此而回来,虽然房子建好的话人都可以住进去,对大家来说都是好事……

\subsection{}

\noindent 4.


\end{document}