\documentclass[12pt]{article}
\usepackage[utf8]{inputenc}
\usepackage[UTF8]{ctex}
\usepackage{biblatex}
\usepackage{amssymb}
\usepackage{latexsym}
\usepackage{amsmath}
\usepackage{cases}
\usepackage{geometry}
\usepackage{graphicx}
\usepackage{float}
\usepackage{listings}
\usepackage{enumerate}
\usepackage{color}
\newcommand{\memo}[1]{\color{red}#1}

\addbibresource{bib.bib}
% \setlength{\parindent}{0em}
\bibliography{bib}
\geometry{a4paper}
% \geometry{a4paper,scale=0.8}
\linespread{1.5}

\begin{document}

\begin{titlepage}
\begin{center}
\linespread{1.2}~\\[1cm]
\linespread{1}
% \includegraphics[width=10cm]{name.pdf}\\[3cm]
\linespread{1.9}\huge {\bfseries 收获与播种}\\[2.5cm]
\linespread{1.9}\Large {\bfseries ----- 对数学家过往的反思\footnotemark 与见证}\\[2.5cm]
\footnotetext{\memo{La traduction des mots "r\'eflexion" et "t\'emoignage" ?}}

\Large 格罗滕迪克~著\\[0.3cm]
\Large fdzk 译\\[4cm] 
\large\today
\end{center}
\end{titlepage}

~\\[4cm]
\begin{center}
    \hspace{2cm} 致父母
\end{center}
\newpage

\tableofcontents



\section{作为序言……}

\section{漫游穿过一件作品 ----- 或者说孩子与母亲}

\subsection{事物的魔力}

\subsection{独处的重要性}

\subsection{内心的冒险 ----- 幻想与见证}

\section{一封信}

\subsection{千页的信}

\hfill 1985年5月

\noindent 1. 我传给你的这些文字,由我的大学细心打印出了有限份样本,它不是单行本,也不是预印本。它的名字,收获与播种,很清楚地说明了这点。我把它寄给你就好像寄一封长长的信 ----- 一封非常私人的信。要是我寄给了你,而非满足于你总有一天会通过书店里卖的书(如果有出版商冒这个险的话)里了解到的(如果你好奇的话),那是因为比起跟别人讲,我更愿意跟你讲。在写这些的时候我不止一次想到你 ----- 必须得说这封信我写了一年多了,并且整个人都投入到其中。这是我赠与你的礼物,而且我写的时候很细心地予你我(每个时刻)曾拥有的能赠与的最好的东西。我不知道这个礼物是否会被接受 ----- 你的答复(与否)会让我明白的……

与此同时,我也将《收获与播种》发给了我数学界所有的同事、朋友们和(前)学生们,我与他们曾在某些时刻有过紧密接触,或者他们曾以这种或那种方式出现在我的反思\footnotemark 中,或指名道姓地或否。有可能你就出现在我的思索中,而且你要是用心而不是只用眼睛和脑袋读的话,肯定能在我没点出你名字的地方认出你自己来。我还把《收获与播种》寄给了其他一些做或者不做科学研究的朋友。
\footnotetext{译注:这里“反思”(réflexion)对应副标题中的“反思”二字。}

你现在在读的,这封向你宣布并呈上“千页的信”(作为前菜……)的“前言信”,也起到序言的作用。写着几行字的时候,序言还没写。另外,《收获与播种》总共有五章(不包括额外的前言)。这里我寄给你第一部分(自负与更新),第二部分(葬礼(1) ----- 或着说中国皇帝的长袍),和第四部分(葬礼(3) ----- 或者说四个opérations)\footnotemark。这些我觉得和你关系比较大。第三部分(葬礼(2) ----- 阴阳的秘诀)无疑是我见证中最私人的部分,同时也比其他部分更具有超出写成之时所处场合的“普遍”价值。在第四部分(四个opérations)里我会时常涉及这部分的内容,但第四部分还是可以单独地看,甚至(很大程度上)可以独立于前三部分阅读\footnotemark。如果读我发的这些能让你答复我(这是我所期望的)并且使你想读剩下的部分,要告诉我。我会很高兴把它们寄给你,只要你的答复让我感受到你是真的感兴趣而非表面上的好奇。

\footnote{我把那些在我的反思中因为种种原因出现、私下里我却不认识的同事单独拎出来。我只给他们寄“四个opérations”(这和他们最相关),还有“第零卷”,包括这封信和《收获与播种》的前言(加上整个前四部分的详细目录)。}
\footnote{通常,你会注意到(“自负与更新”)里的每一节或者接下来三部分的每个“note”都自成一体。他们可以独立于其余部分阅读,就好像我们可以饶有兴致地看一只手、一只脚、一根手指或一根脚趾或者整个身体其他大大小小的部分,却不会忘记这是某个整体的一部分,而正是这个(隐含的)整体赋予了这各部分所有的意义。}




\subsection{收获与播种的诞生 ----- 迅速的回顾}

asda \footnotemark a sd
\footnotetext{asd}


\end{document}